\documentclass[12pt]{article}
\usepackage[T1]{fontenc}
\usepackage{algolrevived}
\usepackage{xcolor}
\definecolor{bg}{cmyk}{0	, 0  , 0   , 0.99}
\color{white}
\pagecolor{bg}
 \usepackage[margin=1in]{geometry}
\begin{document}
	{\centering \huge \bf Ray Optics \par}
	\vspace{20pt}
	\noindent {\bf \large Notes:}
	\vspace{5pt}
	
	\noindent - {\bf Light:} is a combination of perpendicularly oscillating electric and magnetic fields.
	\vspace{5pt}
	
	\noindent - {\bf For Reflection:} incident angle = resulting angle.
	\vspace{5pt}
	
	\noindent {\bf - Refraction:} due to a change in speed of the ray, causes a change in direction - different surfaces have different indexes of refraction (see the law of refraction for an equation).
	\vspace{5pt}
	
	\noindent - {\bf Dispersion:} when different wavelengths of light travelling together refract to a different degree due to the refractive index. Shorter wavelengths experience more significant refraction.
	\vspace{5pt}
	
	\noindent - {\bf Total Internal Reflection:} when light travels from a higher refractive index to a lower one (critical angle).
	\vspace{5pt}
	
	\noindent - {\bf Fiber Optic Cable:} Light Entering at the correct angle bounces along the core without escaping -> If the light enters below the {\bf acceptance angle}, light gets guided by internal reflection. If above, the light escapes.
	\vspace{5pt}
	
	\vspace{20pt}
	\noindent {\bf \large Demo:}
	\vspace{5pt}
	
	
	\noindent - An arduino programmed to send binary characters at set intervals, causing light to flash through the fiber optic cable, to be received by a solar panel (light sensor), and decoded by a second arduino. 
	\vspace{5pt}
	
	
\end{document}